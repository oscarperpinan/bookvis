% Memoir v3.7g required
\settypeblocksize{6.5in}{4.5in}{*}
\setlrmargins{*}{*}{1}
\checkandfixthelayout

\usepackage[english]{babel}
\usepackage[citestyle=authoryear,bibstyle=authoryear,doi=true,url=true]{biblatex}
\let\cite\parencite
\addbibresource{stBib.bib}

\usepackage[usenames,dvipsnames]{xcolor}
\usepackage[T1]{fontenc}
\usepackage[utf8]{inputenc}

\usepackage{url}
\usepackage{hyperref}
\hypersetup{
    bookmarks=true,         % show bookmarks bar?
    bookmarksnumbered=false,
    bookmarksopen=false,
    breaklinks=true,
    backref=true,
    pdftoolbar=true,        % show Acrobat’s toolbar?
    pdfmenubar=true,        % show Acrobat’s menu?
    pdffitwindow=false,     % window fit to page when opened
    pdfstartview={FitH},    % fits the width of the page to the window
    pdftitle={Displaying time series, spatial and space-time data with R},    % title
    pdfauthor={Oscar Perpiñán Lamigueiro},     % author
    pdfcreator={Emacs},   % creator of the document
    pdfproducer={org}, % producer of the document
    pdfnewwindow=true,      % links in new window
    pdfborder={0 0 0},
    colorlinks=true,       % false: boxed links; true: colored links
    linkcolor=black,          % color of internal links
    citecolor=black,        % color of links to bibliography
    filecolor=black,      % color of file links
    urlcolor=Blue           % color of external links 
}

\usepackage{amssymb}
\usepackage{amsmath}
\usepackage{graphicx}
\usepackage{makeidx}
\usepackage{pifont}
\usepackage{fourier}
\usepackage{siunitx}
\usepackage{textcomp}

\usepackage[noprefix]{nomencl}
\renewcommand{\nomlabel}[1]{\textbf{#1}:}
\makenomenclature

\renewcommand{\nomname}{Glossary}

%% Requirement from C&H
\copypagestyle{RuledCH}{Ruled}
\makepsmarks {RuledCH}{
  \nouppercaseheads
  \createmark {chapter} {both} {shownumber} {} {\ }
  \createmark {section} {right} {shownumber} {} {\ }
}
\pagestyle{RuledCH}


\captionnamefont{\scshape}
\usepackage{mathpazo}

\usepackage{memhfixc}
\raggedbottom
\sloppybottom
\clubpenalty=10000
\widowpenalty=10000
\feetbelowfloat

\chapterstyle{ger}

\setlength{\afterchapskip}{35pt}
\maxtocdepth{section}
\setsecnumdepth{subsubsection}
\renewcommand{\topfraction}{0.85}
\renewcommand{\bottomfraction}{0.5}
\renewcommand{\textfraction}{0.15}
\renewcommand{\floatpagefraction}{0.7}

\usepackage{tcolorbox}
\tcbuselibrary{breakable}
\tcbset{colback=gray!5, breakable}

\renewcommand{\prechapterprecis}{%
  \vspace*{\prechapterprecisshift}%
  \begin{tcolorbox}\precisfont}
  
\renewcommand{\postchapterprecis}{\end{tcolorbox}\vfill}

\renewcommand*{\precisfont}{\normalfont}

\usepackage{breakurl}

%Centra las figuras en los flotantes y los enmarca
\makeatletter
\renewenvironment{figure}[1][]{%
     	\@float{figure}%
		\precaption{\rule{\linewidth}{0.4pt}\par}%En las figuras el caption va debajo
		\centering
		  }{%
    	\end@float	
}
\makeatother

\makeatletter
\renewenvironment{table}[1][]{%
      	\@float{table}%
		\postcaption{\rule{\linewidth}{0.4pt}\par}%En las tablas el caption va encima
		\centering
		  }{%
    	\end@float	
}
\makeatother

\renewcommand{\textfloatsep}{10pt}%Espacio entre el flotante y el texto

\usepackage{listings}
\newcommand\indexfunctions[1]{\index{Functions!#1@\texttt{#1}}}
\lstset{
  numberstyle=\tiny\color{Blue},
  framexleftmargin=5mm,
  xleftmargin=\parindent,
  keywordstyle=\color{Blue},
  emphstyle=\color{Blue},
  commentstyle=\color{OliveGreen},
  stringstyle=\color{BrickRed},
  basicstyle=\ttfamily\small,
  index=[1][emph],
  indexstyle=[1]\indexfunctions, %listings genera el index de las funciones que va encontrando de las especificadas en myR.tex
  columns=fullflexible,
  breaklines=true,
  linewidth=\textwidth,
  backgroundcolor=\color{gray!7},
  basewidth={0.5em,0.4em},
  showstringspaces=false,
%  frame=single,
  literate={á}{{\'a}}1 {ñ}{{\~n}}1 {é}{{\'e}}1 {ó}{{\'o}}1 {º}{{\textordmasculine}}1
}

\lstdefinelanguage{myR}{
  language=R,
  alsoletter={.},
  morekeywords={
  data.frame,
  as.data.frame,
  as.numeric,
  read.table,
  read.csv,
  write.csv,
  read.csv2,
  write.csv2,
  as.Date,
  do.call,
  head,
  with,
  within,
  unzip,
  Reduce},
  emph={xyplot,
  ggplot,
  gplot,
  levelplot,
  densityplot,
  rasterTheme,
  plot3D,
  plot3d,
  rglwidget,
  playwidget,
  surface3d,
  rgl.viewpoint,
  movie3d,
  writeWebGL,
  layer,
  glayer,
  geom_point,
  geom_line,
  geom_density,
  theme_bw,
  geom_sf,
  coord_sf,
  geom_raster,
  scale_fill_brewer,
  autoplot,
  zoo,
  read.zoo,
  as.yearmon,
  raster,
  krige,
  spplot,
  sp.lines,
  sp.points,
  sp.polygons,
  sp.pointLabel,
  sp.layout,
  stplot,
  horizonplot,
  hovmoller,
  custom.theme,
  fortify,
  splom,
  melt,
  dcast,
  hexbinplot,
  useOuterStrips,
  simpleTheme,
  st_read,
  st_write,
  st_transform,
  classIntervals,
  findInterval,
  findCols,
  bbox,
  spTransform,
  unionSpatialPolygons,
  spCbind,
  colorRampPalette,
  colorRamp,
  adjustcolor,
  brewer.pal,
  geom_density,
  hillShade,
  terrain,
  cellStats,
  stack,
  SpatialPointsDataFrame,
  SpatialPoints,
  SpatialPolygonsDataFrame,
  SpatialPolygons,
  SpatialLinesDataFrame,
  SpatialLines,
  SpatRaster,
  sf,
  vectorplot,
  streamplot,
  streamTheme,
  extent,
  proj4string,
  projectExtent,
  coordinates,
  brick,
  setZ,
  zApply,
  crop,
  ratify,
  map,
  map2SpatialLines,
  histogram,
  bwplot,
  CRS,
  gIntersection,
  gcIntermediate,
  trellis.device,
  panel.xblocks,
  panel.text,
  panel.number,
  panel.points,
  panel.polygon,
  panel.hexbinplot,
  panel.loess,
  panel.superpose,
  panel.rug,
  diag.panel.splom,
  current.panel.limits,
  panel.link.splom,
  superpose.polygon,
  trellis.par.get,
  trellis.focus,
  grid.ls,
  grid.garnish,
  grid.script,
  grid.export,
  animUnit,
  animValue,
  grid.animate,
  grid.hyperlink,
  grid.raster,
  grid.points,
  grid.text,
  grid.circle,
  grid.rect,
  grid.segments,
  ggpairs,
  direct.label,
  readHTMLTable,
  htmlParse,
  getNodeSet,
  char2dms,
  mapview,
  cubeView,
  plot_ly,
  gvisMotionChart,
  dygraph,
  highchart,
  streamgraph
  }
}


\renewcommand{\lstlistingname}{Code}

\usepackage{fancyvrb}
\DefineVerbatimEnvironment{verbatim}{Verbatim}{%
  fontsize=\small,
  formatcom = {\color{gray!97}}}

\pretitle{\vfill\begin{flushright}\bfseries\scshape\HUGE}
\posttitle{\par\end{flushright}}

\preauthor{\begin{flushright}\large\scshape}
\postauthor{\par\end{flushright}}

\predate{\vfill\begin{flushright}\large\scshape}
\postdate{\par\end{flushright}\vfill}

\makeindex

%% Hyphenation rules for code inside texttt
\usepackage[htt]{hyphenat}
\hyphenation{Spatial-Points}
\hyphenation{Spatial-Pixels}
\hyphenation{Spatial-Grid}
\hyphenation{Spatial-Lines}
\hyphenation{Spatial-Polygons}

\hyphenation{Spatial-Points-Data-Frame}
\hyphenation{Spatial-Lines-Data-Frame}
\hyphenation{Spatial-Polygons-Data-Frame}

\hyphenation{Raster-Layer}
\hyphenation{Raster-Stack}
\hyphenation{Raster-Brick}

\hyphenation{read-Shape-Poly}
\hyphenation{read-Shape-Points}
\hyphenation{read-Shape-Lines}
\hyphenation{write-Points-Shape}
\hyphenation{write-Lines-Shape}
\hyphenation{write-Poly-Shape}
\hyphenation{union-Spatial-Polygons}

\hyphenation{Open-Street-Maps}

\hyphenation{Java-Script}

\hyphenation{space-time}

%% Requirement from C&H
\setcounter{page}{5}
