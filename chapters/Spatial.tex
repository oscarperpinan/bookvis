\part{Spatial Data}
\label{cha:Spatial}

\chapter{Displaying Spatial Data: Introduction}
\label{cha:spatialIntro}

Spatial data (also known as geospatial data) are directly or
indirectly referenced to a location on the surface of the Earth. Their
spatial reference is composed of coordinate values and a system of
reference for these coordinates. Spatial data are often accessed,
manipulated, or analyzed through Geographic Information Systems (GIS).

Real objects represented by GIS data can be divided into two
abstractions: discrete objects (e.g., a road or a river)
represented with vector data (points, lines, and polygons), and
continuous fields (such as elevation or solar radiation)
represented with raster data. The \texttt{sp} package is the
preferred option to use vector data in \textsf{R}, and the
\texttt{raster} package is the choice for raster
data\footnote{Although \texttt{sp} and \texttt{raster} are the
  most important packages, there are an increasing number of
  packages designed to work with spatial data. They are summarized
  in the corresponding CRAN Task View. Read Section
  \ref{cha:further-reading-spatial} for details.}.

This part exposes several examples where vector and raster data
are displayed to show geographic location of features and physical
landscape features of a place (reference and physical maps,
Chapter \ref{cha:refer-phys-maps}) or a specific variable in the
context of a geographic reference (thematic maps, Chapter
\ref{cha:thematicMaps}). These examples make use of several datasets
(available at the book website) described in Chapter
\ref{cha:dataSpatial}.


\section{Packages}
\label{sec:spatial-packages}

The CRAN Tasks View ``Analysis of Spatial
Data''\footnote{\url{http://CRAN.R-project.org/view=Spatial}}
summarizes the packages for reading, vizualizing, and analyzing
spatial data. This section provides a brief introduction to
\texttt{sp}, \texttt{raster}, \texttt{rasterVis}, \texttt{maptools},
\texttt{rgdal}, \texttt{gstat}, and \texttt{maps}. Most of the
information has been extracted from their vignettes, webpages, and help
pages. You should read them for detailed information.

\subsection{sp}
\label{sec:sp}

\index{Packages!sp@\texttt{sp}}

The \texttt{sp} package \cite{Pebesma.Bivand2005} provides classes and
methods for dealing with spatial data in \textsf{R}. The spatial data
classes implemented are points (\texttt{SpatialPoints}), grids
(\texttt{SpatialPixels} and \texttt{SpatialGrid}), lines
(\texttt{Line}, \texttt{Lines} and \texttt{SpatialLines}), rings, and
polygons (\texttt{Polygon}, \texttt{Polygons}, and
\texttt{SpatialPolygons}), each of them without data or with data (for
example, \texttt{SpatialPointsDataFrame} or
\texttt{SpatialLinesDataFrame}).

Selecting, retrieving, or replacing certain attributes in spatial
objects with data is done using standard methods:
\begin{itemize}
\item \texttt{[} selects rows (items) and columns in the
  \texttt{data.frame}.
\item \texttt{[[} selects a column from the \texttt{data.frame}
\item \texttt{[[<-} assigns or replaces values to a column in the
  \texttt{data.frame}.
\end{itemize}

A number of spatial methods are available for the classes in \texttt{sp}:
\begin{itemize}
\item \texttt{coordinates(object) <- value} sets spatial coordinates
  to create spatial data. It promotes a \texttt{data.frame} into a
  \texttt{SpatialPointsDataFrame}. \emph{value} may be specified by a
  formula, a character vector, or a numeric matrix or
  \texttt{data.frame} with the actual coordinates.
\item \texttt{coordinates(object, ...)} returns a matrix with the
  spatial coordinates. If used with \texttt{SpatialPolygons} it
  returns a matrix with the centroids of the polygons.
\item \texttt{bbox} returns a matrix with the coordinates bounding
  box.
\item \texttt{proj4string(object)} and \texttt{proj4string(object) <-
    value} retrieve or set projection attributes on spatial classes.
\item \texttt{spTransform} transforms from one coordinate reference
  system (geographic projection) to another (requires package
  \texttt{rgdal}).
\item \texttt{spplot} plots attributes combined with spatial data:
  Points, lines, grids, polygons.
\end{itemize}

\subsection{raster}
\label{sec:raster}

\index{Packages!raster@\texttt{raster}}

The \texttt{raster} package \cite{Hijmans2013} has functions for
creating, reading, manipulating, and writing raster data. The package
provides general raster data manipulation functions. The package also
implements raster algebra and most functions for raster data
manipulation that are common in Geographic Information Systems (GIS).

The raster package can work with raster datasets stored on disk if
they are too large to be loaded into memory. The package can work with
large files because the objects it creates from these files only
contain information about the structure of the data, such as the
number of rows and columns, the spatial extent, and the filename, but
it does not attempt to read all the cell values in memory. In
computations with these objects, the data are processed in chunks.

The package defines a number of \texttt{S4}
classes. \texttt{RasterLayer}, \texttt{RasterBrick}, and
\texttt{RasterStack} are the most important:
\begin{itemize}
\item A \texttt{RasterLayer} object represents single-layer (variable)
  raster data. It can be created with the function
  \texttt{raster}. This function is able to create a
  \texttt{RasterLayer} from another object, including another
  \texttt{Raster*} object, or from a \texttt{SpatialPixels*} and
  \texttt{SpatialGrid*} object, or even a matrix. In addition, it can
  create a \texttt{RasterLayer} reading data from a file. The
  \texttt{raster} package can use raster files in several formats,
  some of them via the \texttt{rgdal} package. Supported formats for
  reading include GeoTIFF, ESRI, ENVI, and ERDAS.

\item \texttt{RasterBrick} and \texttt{RasterStack} are classes for
  multilayer data. A \texttt{RasterStack} is a list of
  \texttt{RasterLayer} objects with the same spatial extent and
  resolution. It can be formed with a collection of files in different
  locations or even mixed with \texttt{RasterLayer} objects that only
  exist in memory.  A \texttt{RasterBrick} is truly a multilayered
  object, and processing it can be more efficient than processing a
  \texttt{RasterStack} representing the same data.

\end{itemize}

The \texttt{raster} package defines a number of methods for raster
algebra with \texttt{Raster*} objects: arithmetic operators, logical
operators, and functions such as \texttt{abs}, \texttt{round},
\texttt{ceiling}, \texttt{floor}, \texttt{trunc}, \texttt{sqrt},
\texttt{log}, \texttt{log10}, \texttt{exp}, \texttt{cos},
\texttt{sin}, \texttt{max}, \texttt{min}, \texttt{range},
\texttt{prod}, \texttt{sum}, \texttt{any}, and \texttt{all}. In these
functions, \texttt{Raster*} objects can be mixed with numbers.

There are several functions to modify the content or the spatial
extent of \texttt{Raster*} objects, or to combine \texttt{Raster*}
objects:
\begin{itemize}
\item The \texttt{crop} function takes a geographic subset of a larger
  \texttt{Raster*} object. \texttt{trim} crops a \texttt{RasterLayer}
  by removing the outer rows and columns that only contain \texttt{NA}
  values. \texttt{extend} adds new rows and/or columns with
  \texttt{NA} values.
\item The \texttt{merge} function merges two or more \texttt{Raster*}
  objects into a single new object.
\item \texttt{projectRaster} transforms values of a \texttt{Raster*}
  object to a new object with a different coordinate reference system.
\item With \texttt{overlay}, multiple \texttt{Raster*} objects can be
  combined (for example, multiply them).
\item \texttt{mask} removes all values from one layer that are
  \texttt{NA} in another layer, and \texttt{cover} combines two layers
  by taking the values of the first layer except where these are
  \texttt{NA}.
\item \texttt{calc} computes a function for a \texttt{Raster*}
  object. With \texttt{RasterLayer} objects, another
  \texttt{RasterLayer} is returned. With multilayer objects the result
  depends on the function: With a summary function (\texttt{sum},
  \texttt{max}, etc.),  \texttt{calc} returns a \texttt{RasterLayer}
  object, and a \texttt{RasterBrick} object otherwise.
\item \texttt{stackApply} computes summary layers for subsets of a
  \texttt{RasterStack} or \texttt{RasterBrick}.
\item \texttt{cut} and \texttt{reclassify} replace ranges of values
  with single values.
\item \texttt{zonal} computes zonal statistics, that is, summarizes a
  \texttt{Raster*} object using zones (areas with the same integer
  number) defined by another \texttt{RasterLayer}.
\end{itemize}


\subsection{rasterVis}
\label{sec:rasterVis}
\index{Packages!rasterVis@\texttt{rasterVis}}

The \texttt{rasterVis} package \cite{Perpinan.Hijmans2013} complements
the \texttt{raster} package, providing a set of methods for enhanced
visualization and interaction. This package defines visualization
methods (\texttt{levelplot}) for quantitative data and categorical
data, both for univariate and multivariate rasters.

It also includes several methods in the frame of the Exploratory Data
Analysis approach: scatterplots with \texttt{xyplot}, histograms and
density plots with \texttt{histogram} and \texttt{densityplot}, violin
and boxplots with \texttt{bwplot}, and a matrix of scatterplots with
\texttt{splom}.

On the other hand, this package is able to display vector fields using
arrows, \texttt{vectorplot}, or with streamlines
\cite{Wegenkittl.Groeller1997}, \texttt{streamplot}. In this last
method, for each point, \emph{droplet}, of a jittered regular grid, a
short streamline portion, \emph{streamlet}, is calculated by
integrating the underlying vector field at that point. The main color
of each streamlet indicates local vector magnitude (slope). Streamlets
are composed of points whose sizes, positions, and color degradation
encode the local vector direction (aspect).


\subsection{maptools}
\label{sec:maptools}
\index{Packages!maptools@\texttt{maptools}}

The \texttt{maptools} package \cite{Bivand.Lewin-Koh2013} provides a
set of tools for manipulating and reading geographic data, in
particular ESRI (Environmental Systems Research Institute)
shapefiles. The package also provides interface wrappers for
exchanging spatial objects with packages such as PBSmapping, spatstat,
maps, RArcInfo, Stata tmap, WinBUGS, Mondrian, and others. The main
functions in the context of this book are

\begin{itemize}

\item \texttt{readShapePoints} reads data from a points shapefile into
  a \texttt{SpatialPointsDataFrame} object.

\item \texttt{writePointsShape} writes data from a
  \texttt{SpatialPointsDataFrame} object to a shapefile.

\item \texttt{readShapeLines} reads data from a line shapefile
  into a \texttt{SpatialLinesDataFrame} object.

\item \texttt{writeLinesShape} writes data from a
  \texttt{SpatialLinesDataFrame} object to a shapefile.

\item \texttt{readShapePoly} reads data from a polygon shapefile into
  a \texttt{SpatialPolygonsDataFrame} object.

\item \texttt{writePolyShape} writes data from a
  \texttt{SpatialPolygonsDataFrame} object to a shapefile.

\item \texttt{map2SpatialPolygons} and \texttt{map2SpatialLines} may
  be used to convert map objects returned by the \texttt{map} function
  in the \texttt{maps} package to the classes defined in the
  \texttt{sp} package.

\item \texttt{spCbind} provides cbind-like methods for
  \texttt{Spatial*DataFrame} and \texttt{data.frame} objects.
 
\end{itemize}

The topology operations on geometries performed by this package (for
example, \texttt{unionSpatialPolygons} ) use the package
\texttt{rgeos}, an interface to the Geometry Engine Open Source
(GEOS)\footnote{\url{http://trac.osgeo.org/geos/}}.


\subsection{rgdal}
\label{sec:rgdal}
\index{Packages!rgdal@\texttt{rgdal}}

The \texttt{rgdal} package \cite{Bivand.Keitt.ea2013} provides
bindings to the Geospatial Data Abstraction Library
(GDAL)\footnote{\url{http://www.gdal.org/}}. With \texttt{readOGR} and
\texttt{readGDAL}, both GDAL raster and OGR vector map data can be
imported into \textsf{R}, and GDAL raster data and OGR vector data can
be exported with \texttt{writeGDAL} and \texttt{writeOGR}.

In addition, this package provides access to projection and
transformation operations from the PROJ.4
library\footnote{\url{https://trac.osgeo.org/proj/}}. This package
implements several \texttt{spTransform} methods providing
transformation between datums and conversion between projections using
PROJ.4 projection arguments.


\subsection{gstat}
\label{sec:gstat}
\index{Packages!gstat@\texttt{gstat}}

The \texttt{gstat} package \cite{Pebesma2004} provides functions for
geostatistical modeling, prediction, and simulation, including
variogram modeling and simple, ordinary, universal, and external drift
kriging.

Most of the functionality of this package is beyond the scope of this
book. However, some functions must be mentioned:
\begin{itemize}
\item \texttt{variogram} calculates the sample variogram from data, or
  for the residuals if a linear model is given. \texttt{vgm} generates
  a variogram and \texttt{fit.variogram} fit ranges and/or sills from
  a variogram model to a sample variogram.
\item \texttt{krige} is the function for simple, ordinary or universal
  kriging. \texttt{gstat} is the function for univariate or
  multivariate geostatistical prediction.
\end{itemize}

\subsection{maps}
\label{sec:maps}
\index{Packages!maps@\texttt{maps}}
\index{Packages!mapproj@\texttt{mapproj}}
\index{Packages!mapdata@\texttt{mapdata}}

The \texttt{maps} \cite{Becker.Wilks.ea2013}, \texttt{mapdata}
\cite{Becker.Wilks.ea2013b}, and \texttt{mapproj}
\cite{McIlroy.Brownrigg.ea2013} packages are useful to draw or create
geographical maps. \texttt{mapdata} contains higher resolution
databases, and \texttt{mapproj} converts latitude/longitude
coordinates into projected coordinates.


\section{Further Reading}
\label{cha:further-reading-spatial}

\begin{itemize}

\item \cite{Slocum.McMaster.ea2005} and \cite{Dent.Torguson.ea2008}
  are comprehensive books on thematic cartography and
  geovisualization. They include chapters devoted to data
  classification, scales, map projections, color theory, typography,
  and proportional symbol, choropleth, dasymetric, isarithmic, and
  multivariate mapping. Several resources are available at their
  accompanying
  websites\footnote{\url{http://www.pearsonhighered.com/slocum3e/} and
    \url{http://highered.mcgraw-hill.com/sites/0072943823/}}.

\item \cite{Bivand.Pebesma.ea2008} is the essential reference to work
  with spatial data in \textsf{R}. R. Bivand and E. Pebesma are the
  authors of the fundamental \texttt{sp} package, and they are the
  authors or maintainers of several important packages such as
  \texttt{gstat}, for geostatistical modeling, prediction, and
  simulation, \texttt{rgdal}, \texttt{rgeos} and
  \texttt{maptools}. Chapter 3 is devoted to the visualization of
  spatial data. Code, figures, and data of the book are available at
  the accompanying website\footnote{\url{http://www.asdar-book.org/}}.

\item \cite{Hengl2009} is an open-access book with seven spatial
  data analysis exercises. The author is the creator and
  maintainer of the Spatial-Analyst
  webpage\footnote{\url{http://spatial-analyst.net}}.

\item The CRAN Tasks View ``Analysis of Spatial
  Data''\footnote{\url{http://CRAN.R-project.org/view=Spatial}}
  summarizes the packages for reading, vizualizing, and analyzing
  spatial data.  The packages in development published at R-Forge are
  listed in the ``Spatial Data \& Statistics'' topic
  view\footnote{\url{http://r-forge.r-project.org/softwaremap/trove_list.php?form_cat=353}}.
  The R-SIG-Geo mailing
  list\footnote{\url{https://stat.ethz.ch/mailman/listinfo/R-SIG-Geo/}}
  is a powerful resource for obtaining help.

\item The ``Spatial
  Analysis''\footnote{\url{http://spatialanalysis.co.uk/map-gallery/}}
  and ``Kartograph''\footnote{\url{http://kartograph.org/}} webpages
  publish a variety of beautiful visualization examples.

\end{itemize}


\chapter{Thematic Maps}
\label{cha:thematicMaps}

A thematic map focuses on a specific theme or variable, commonly using
geographic data such as coastlines, boundaries, and places as points
of reference for the variable being mapped. These maps provide
specific information about particular locations or areas (proportional
symbol mapping and choropleth maps) and information about spatial
patterns (isarithmic and raster maps). The following sections illustrate
the \textsf{R} code you need to produce these maps, with a final
section devoted to the visualization of vector fields.

\input{Spatial/bubble}
\input{Spatial/choropleth}
\input{Spatial/raster}
\input{Spatial/vector}

\chapter{Reference and Physical Maps}
\label{cha:refer-phys-maps}

A reference map focuses on the geographic location of features. In
these maps, cities are named and major transport routes are
identified. In addition, natural features such as rivers and mountains are
named, and elevation is shown using a simple color shading.  

A physical map shows the physical landscape features of a
place. Mountains and elevation changes are usually shown with
different colors and shades to show relief, using green to show
lower elevations and browns for high elevations.

This chapter details how to create a reference map of a northern
region of Spain using data from OpenStreetMap and a physical map of
Brazil with data from different sources.

\input{Spatial/physical}



Although I was born in Madrid, Galicia (north of Spain) is a very
special region for me. More precisely, Cedeira and Valdoviño offer
a wonderful combination of wild sea, secluded beaches and
forests. I will show you a map of this marvelous places. 

The first step is to acquire information from the OpenStreetMap
project. There are several packages to extract data from this
service but, while most of them only provide already rendered
raster images, the \texttt{osmar} package enables the use of the raw data
with classes from the packages \texttt{sp} and \texttt{igraph}.

The \texttt{get\_osm} function retrieves a region defined by \texttt{corner\_bbox}
using the OSM API.


\lstset{language=R}
\begin{lstlisting}
library('osmar')

api <- osmsource_api()
ymax <- 43.7031
ymin <- 43.6181
xmax <- -8.0224
xmin <- -8.0808
box <- corner_bbox(xmin, ymin, xmax, ymax)
cedeira <- get_osm(box, source=api, full=TRUE)
\end{lstlisting}


The \texttt{cedeira} object includes three main components: nodes, ways
and relations. 
  

\lstset{language=R}
\begin{lstlisting}
summary(cedeira)
summary(cedeira$nodes)
\end{lstlisting}

These components can be accessed with the functions \texttt{find}, \texttt{subset}, \texttt{way},
\texttt{node}, \texttt{relation} and \texttt{tags}. Thus, the different kinds of roads
can be obtained using \texttt{way} and \texttt{tags} with the appropiate
tag. 


\lstset{language=R}
\begin{lstlisting}
idxHighways <- find(cedeira, way(tags(k=='highway')))
highways <- subset(cedeira, way_ids=idxHighways)
idxStreets <- find(highways, way(tags(v=='residential')))
idxPrimary <- find(highways, way(tags(v=='primary')))
idxSecondary <- find(highways, way(tags(v=='secondary')))
idxTertiary <- find(highways, way(tags(v=='tertiary')))
idxOther <- find(highways,
                 way(tags(v=='unclassified' |
                          v=='footway' |
                          v=='steps')))
\end{lstlisting}

The result of \texttt{find} is the index of each element. The
correspondent spatial object is extracted with \texttt{find\_down} and
\texttt{subset}, and can be converted to a class defined by the \texttt{sp}
package with \texttt{as\_sp}. The next \texttt{spFromOSM} function encodes the
procedure, and extracts the \texttt{SpatialLines} which represent each
type of road.


\lstset{language=R}
\begin{lstlisting}
spFromOSM <- function(source, index, type='lines'){
  idx <- find_down(source, index)
  obj <- subset(source, ids=idx)
  objSP <- as_sp(obj, type)
  }

streets <- spFromOSM(cedeira, way(idxStreets))
primary <- spFromOSM(cedeira, way(idxPrimary))
secondary <- spFromOSM(cedeira, way(idxSecondary))
tertiary <- spFromOSM(cedeira, way(idxTertiary))
other <- spFromOSM(cedeira, way(idxOther))
\end{lstlisting}
  
A similar procedure can be applied to construct a \texttt{SpatialPoints}
object with the collection of places with name:

\lstset{language=R}
\begin{lstlisting}
idxPlaces <- find(cedeira, node(tags(k=='name')))
places <- spFromOSM(cedeira, node(idxPlaces), 'points')

nms <- subset(cedeira$nodes$tags, subset=(k=='name'), select=c('id', 'v'))
ord <- match(idxPlaces, nms$id)
nms <- nms[ord,]
places$name <- nms$v[ord]
\end{lstlisting}

The second step is to produce a hill shade layer. This layer can
be computed from the slope and aspect layers derived from a
Digital Elevation Model. The DEM for this region is available at
the Geonetwork-SECAD service from the Universidad de Extremadura
and can be read with \texttt{raster}:

\lstset{language=R}
\begin{lstlisting}
## Galicia DEM
## http://ide.unex.es/geonetwork/srv/es/main.search?any=MDE_Galicia
## http://ide.unex.es:8180/geonetwork/srv/es/resources.get?id=21&fname=dem_gal.7z&access=private

old <- tempdir()
download.file('http://ide.unex.es:8180/geonetwork/srv/es/resources.get?id=21&fname=dem_gal.7z&access=private', 'dem_gal.7z')
unzip('dem_gal.7z')
demGalicia <- raster('dem_gal.asc')
setwd(old)
\end{lstlisting}



The \texttt{slope} and \texttt{aspect} layers are computed with the \texttt{terrain}
function, and the hill shade layer is derived with these layers
for a fixed sun position. Previously, the useful region of the DEM
raster is extracted with the \texttt{crop}:


\lstset{language=R}
\begin{lstlisting}
cedeiraSP <- as_sp(cedeira, 'points')
projCedeira <- projection(cedeiraSP)
##extCedeira <- bbox(cedeiraSP) 
## or summary(cedeira$nodes)$bbox
extCedeira <- extent(-8.15, -7.95, 43.6, 43.75)
demCedeira <- crop(demGalicia, extCedeira)
projection(demCedeira) <- projCedeira
demCedeira[demCedeira <= 0] <- NA

slope <- terrain(demCedeira, 'slope')
aspect <- terrain(demCedeira, 'aspect')
hsCedeira <- hillShade(slope=slope, aspect=aspect,
                       angle=20, direction=30)
\end{lstlisting}

And finally, the third step is to display the different layers of
information in correct order: the hill shade layer, the roads, and
the places (points and labels).
  

\lstset{language=R}
\begin{lstlisting}
##Auxiliary function to display the roads. A thicker black line in
##the background and a thinner one with an appropiate color.
sp.road <- function(line, lwd=5, blwd=7,
                    col='indianred1', bcol='black'){
  sp.lines(line, lwd=blwd, col=bcol)
  sp.lines(line, lwd=lwd, col=col)
}

## The background color of the panel is set to blue to represent the sea
myTheme <- GrTheme()
myTheme$panel.background$col = 'skyblue3'



levelplot(hsCedeira, par.settings=myTheme, margin=FALSE, colorkey=FALSE)+
  layer(sp.road(streets, lwd=1, blwd=2, col='white')) +
  layer(sp.road(other, lwd=2, blwd=3, col='white')) +
  layer(sp.road(tertiary, lwd=3, blwd=4, col='palegreen')) +
  layer(sp.road(secondary, lwd=4, blwd=6, col='midnightblue')) +
  layer(sp.road(primary, col='indianred1')) +
  layer(sp.points(places, pch=19, col='black', cex=0.6, alpha=0.5)) +
  layer(sp.pointLabel(places, labels=places$name,
                      fontfamily = 'Palatino', 
                      cex=0.6, col='black'))
\end{lstlisting}

\includegraphics[width=.9\linewidth]{figs/cedeiraOsmar.pdf}






\chapter{About the Data}
\label{cha:dataSpatial}


\section{Air Quality in Madrid}
\label{sec-1}
\label{sec:airQualityData}

Air pollution is harmful to health and contributes to respiratory and
cardiac diseases, and has a negative impact on natural ecosystems,
agriculture, and the built environment. In Spain, the principal
pollutants are particulate matter (PM), tropospheric ozone, nitrogen
dioxide, and environmental noise\footnote{\url{http://www.eea.europa.eu/soer/countries/es/}}.

The surveillance system of the Integrated Air Quality system of the
Madrid City Council consists of twenty-four remote stations, equipped
with analyzers for gases (NO\_\{X\}, CO, ozone, BT\_\{X\}, HCs, SO\_\{2\}) and
particles (PM10, PM2.5), which measure pollution in different areas of
the urban environment. In addition, many of the stations also include
sensors to provide meteorological data.

The detailed information of each measuring station can be retrieved
from its own webpage defined by its station code.
\lstset{language=R,numbers=none}
\begin{lstlisting}
## codeStations.csv is extracted from the document
## http://www.mambiente.munimadrid.es/opencms/export/sites/default/calaire/Anexos/INTPHORA-DIA.pdf,
## table of page 3.

codEstaciones <- read.csv2('data/codeStations.csv')
codURL <- as.numeric(substr(codEstaciones$Codigo, 7, 8))

## The information of each measuring station is available at its own webpage, defined by codURL
URLs <- paste('http://www.mambiente.munimadrid.es/opencms/opencms/calaire/contenidos/estaciones/estacion', codURL, '.html', sep='')
\end{lstlisting}

\subsection{\floweroneleft Data Arrangement}
\label{sec-1-1}
The station webpage includes several tables that can be extracted with
the \texttt{readHTMLTable} function of the \texttt{XML} package.  The longitude and
latitude are included in the second table. The \texttt{ub2dms} function
cleans this table and converts the strings to the \texttt{DMS} class defined
by the \texttt{sp} package to represent degrees, minutes, and decimal
seconds.

\index{Web scraping}
\index{Packages!XML@\texttt{XML}}
\index{Packages!sp@\texttt{sp}}
\index{readHTMLTable@\texttt{readHTMLTable}}
\index{lapply@\texttt{lapply}}
\index{char2dms@\texttt{char2dms}}

\lstset{language=R,numbers=none}
\begin{lstlisting}
library(XML)
library(sp)

## Access each webpage, retrieve tables and extract long/lat data
coords <- lapply(URLs, function(est){
  tables <- readHTMLTable(est)
  location <- tables[[2]]
  ## Clean the table content and convert to dms format
  ub2dms <- function(x){
    ch <- as.character(x)
    ch <- sub(',', '.', ch) 
    ch <- sub('O', 'W', ch) ## Some stations use "O" instead of "W"
    as.numeric(char2dms(ch, "º", "'", "'' "))
  }
  long <- ub2dms(location[2,1])
  lat <- ub2dms(location[2,2])
  alt <- as.numeric(sub(' m.', '', location[2, 3]))

  coords <- data.frame(long=long, lat=lat, alt=alt)

  coords
})

airStations <- cbind(codEstaciones, do.call(rbind, coords))

## The longitude of "El Pardo" station is wrong (positive instead of negative)
airStations$long[22] <- -airStations$long[22]

write.csv2(airStations, file='data/airStations.csv')
\end{lstlisting}

The 2011 air pollution data are available upon request from the Madrid
City Council webpage\footnote{\url{http://www.mambiente.munimadrid.es/opencms/opencms/calaire/consulta/descarga.html}} and at the \texttt{data} folder of the book
repository. The structure of the file is documented in the
INTPHORA-DIA document\footnote{\url{http://www.mambiente.munimadrid.es/opencms/export/sites/default/calaire/Anexos/INTPHORA-DIA.pdf}}. The \texttt{readLines} function reads the file
and a \texttt{lapply} loop processes each line. The result is stored in the
file \texttt{airQuality.csv}

\index{String manipulation}
\index{readLines@\texttt{readLines}}
\index{lapply@\texttt{lapply}}
\index{do.call@\texttt{do.call}}
\index{substr@\texttt{substr}}
\index{gregexpr@\texttt{gregexpr}}
\index{strsplit@\texttt{strsplit}}
\index{gsub@\texttt{gsub}}

\lstset{language=R,numbers=none}
\begin{lstlisting}
## Fill in the form at
## http://www.mambiente.munimadrid.es/opencms/opencms/calaire/consulta/descarga.html
## to receive the Diarios11.zip file.
unzip('data/Diarios11.zip')
rawData <- readLines('data/Datos11.txt')
## This loop reads each line and extracts fields as defined by the
## INTPHORA file:
## http://www.mambiente.munimadrid.es/opencms/export/sites/default/calaire/Anexos/INTPHORA-DIA.pdf
datos11 <- lapply(rawData, function(x){
  codEst <- substr(x, 1, 8)
  codParam <- substr(x, 9, 10)
  codTec <- substr(x, 11, 12)
  codPeriod <- substr(x, 13, 14)
  month <- substr(x, 17, 18)
  dat <- substr(x, 19, nchar(x))
  ## "N" used for impossible days (31st April)
  idxN <- gregexpr('N', dat)[[1]]
  if (idxN==-1) idxN <- numeric(0)
  nZeroDays <- length(idxN)
  day <- seq(1, 31-nZeroDays)
  ## Substitute V and N with ";" to split data from different days
  dat <- gsub('[VN]+', ';', dat)
  dat <- as.numeric(strsplit(dat, ';')[[1]])
  ## Only data from valid days
  dat <- dat[day]
  res <- data.frame(codEst, codParam, ##codTec, codPeriod,
		    month, day, year=2011,
		    dat)
  })
datos11 <- do.call(rbind, datos11)
write.csv2(datos11, 'data/airQuality.csv')
\end{lstlisting}
\subsection{Combine Data and Spatial Locations}
\label{sec-1-2}
Our next step is to combine the data and spatial information. The
locations are contained in \texttt{airStations}, a \texttt{data.frame} that is
converted to an \texttt{SpatialPointsDataFrame} object with the \texttt{coordinates}
method.

\index{Data!Air quality in Madrid}
\index{Packages!sp@\texttt{sp}}
\index{read.csv2@\texttt{read.csv2}}

\lstset{language=R,numbers=none}
\begin{lstlisting}
library(sp)

## Spatial location of stations
airStations <- read.csv2('data/airStations.csv')
coordinates(airStations) <- ~ long + lat
## Geographical projection
proj4string(airStations) <- CRS("+proj=longlat +ellps=WGS84 +datum=WGS84")
\end{lstlisting}

On the other hand, the \texttt{airQuality} \texttt{data.frame} comprises the air
quality daily measurements. We will retain only the $NO_2$ time
series.
\lstset{language=R,numbers=none}
\begin{lstlisting}
## Measurements data
airQuality <- read.csv2('data/airQuality.csv')
## Only interested in NO2 
NO2 <- airQuality[airQuality$codParam==8, ]
\end{lstlisting}

We will represent each station using aggregated values (mean, median,
and standard deviation) computed with \texttt{aggregate}:

\index{aggregate@\texttt{aggregate}}

\lstset{language=R,numbers=none}
\begin{lstlisting}
NO2agg <- aggregate(dat ~ codEst, data=NO2,
		    FUN = function(x) {
			c(mean=signif(mean(x), 3),
			  median=median(x),
			  sd=signif(sd(x), 3))
			})
NO2agg <- do.call(cbind, NO2agg)
NO2agg <- as.data.frame(NO2agg)
\end{lstlisting}

The aggregated values (a \texttt{data.frame}) and the spatial information (a
\texttt{SpatialPointsDataFrame}) are combined with the \texttt{spCbind} method from
the \texttt{maptools} package to create a new
\texttt{SpatialPointsDataFrame}. Previously, the \texttt{data.frame} is reordered by
matching against the shared key column (\texttt{airStations\$Codigo} and
\texttt{NO2agg\$codEst}):

\index{Packages!maptools@\texttt{maptools}}
\index{aggregate@\texttt{aggregate}} \index{match@\texttt{match}}
\index{spCbind@\texttt{spCbind}}

\lstset{language=R,numbers=none}
\begin{lstlisting}
library(maptools)
## Link aggregated data with stations to obtain a SpatialPointsDataFrame.
## Codigo and codEst are the stations codes
idxNO2 <- match(airStations$Codigo, NO2agg$codEst)
NO2sp <- spCbind(airStations[, c('Nombre', 'alt')], NO2agg[idxNO2, ])
save(NO2sp, file='data/NO2sp.RData')
\end{lstlisting}

\section{Spanish General Elections}
\label{sec-2}
The results from the 2011 Spanish general elections\footnote{\url{http://en.wikipedia.org/wiki/Spanish_general_election_2011}} are
available from the Ministry webpage\footnote{\url{http://www.infoelectoral.mir.es/docxl/04_201105_1.zip}} and at the \texttt{data} folder of
the book repository. Each region of the map will represent the
percentage of votes (\texttt{pcMax}) obtained by the predominant political
option (\texttt{whichMax}) at the corresponding municipality.  Only four
groups are considered: the two main parties (\texttt{PP} and \texttt{PSOE}), the
abstention results (\texttt{ABS}), and the remaining parties (\texttt{OTH}). Each
region will be identified by the \texttt{PROVMUN} code.

\index{apply@\texttt{apply}}
\index{sprintf@\texttt{sprintf}}

\lstset{language=R,numbers=none}
\begin{lstlisting}
dat2011 <- read.csv('data/GeneralSpanishElections2011.gz')

census <- dat2011$Total.censo.electoral
validVotes <- dat2011$Votos.válidos
## Election results per political party and municipality
votesData <- dat2011[, 12:1023]
## Abstention as an additional party
votesData$ABS <- census - validVotes
## Winner party at each municipality
whichMax <- apply(votesData,  1, function(x)names(votesData)[which.max(x)])
## Results of the winner party at each municipality
Max <- apply(votesData, 1, max)
## OTH for everything but PP, PSOE and ABS
whichMax[!(whichMax %in% c('PP',  'PSOE', 'ABS'))] <- 'OTH'
## Percentage of votes with the electoral census
pcMax <- Max/census * 100

## Province-Municipality code. sprintf formats a number with leading zeros.
PROVMUN <- with(dat2011, paste(sprintf('%02d', Código.de.Provincia),
			       sprintf('%03d', Código.de.Municipio),
			       sep=""))

votes2011 <- data.frame(PROVMUN, whichMax, Max, pcMax)
write.csv(votes2011, 'data/votes2011.csv', row.names=FALSE)
\end{lstlisting}

\section{CM SAF}
\label{sec-3}
\label{sec:CMSAF}

The Satellite Application Facility on Climate Monitoring (CM SAF) is a
joint venture of the Royal Netherlands Meteorological Institute, the
Swedish Meteorological and Hydrological Institute, the Royal
Meteorological Institute of Belgium, the Finnish Meteorological
Institute, the Deutscher Wetterdienst, Meteoswiss, and the UK
MetOffice, along with collaboration of the European Organization for
the Exploitation of Meteorological Satellites (EUMETSAT)
\cite{CMSAF}. The CM-SAF was funded in 1992 to generate and store
monthly and daily averages of meteorological data measured in a
continuous way with a spatial resolution of $\ang{0.03}$ (15
kilometers). The CM SAF provides two categories of data: operational
products and climate data. The operational products are built on data
that are validated with on-ground stations and then is provided in
near-real-time to develop variability studies in diurnal and seasonal
time scales. However, climate data are long-term data series to assess
inter-annual variability \cite{Posselt.Mueller.ea2012}.

In this chapter we will display the annual average of the shortwave
incoming solar radiation product (SIS) incident over Spain during
2008, computed from the monthly means of this variable. SIS collates
shortwave radiation ($0.2$ to $\SI{4}{\micro\meter}$ wavelength range)
reaching a horizontal unit Earth surface obtained by processing
information from geostationary satellites (METEOSAT) and also from
polar satellites (MetOp and NOAA) \cite{Schulz.Albert.ea2009} and then
validated with high-quality on-ground measurements from the Baseline
Surface Radiation Network (BSRN)\footnote{\url{http://www.bsrn.awi.de/en/home/}}.

The monthly means of SIS are available upon request from the CM SAF
webpage \cite{Posselt.Muller.ea2011} and at the \texttt{data} folder of the
book repository. Data from CM-SAF is published as raster files. The
\texttt{raster} package provides the \texttt{stack} function to read a set of files
and create a \texttt{RasterStack} object, where each layer stores the content
of a file. Therefore, the twelve raster files of monthly averages
produce a \texttt{RasterStack} with twelve layers.

\index{Packages!raster@\texttt{raster}}
\index{stack@\texttt{stack}}

\lstset{language=R,numbers=none}
\begin{lstlisting}
library(raster)

tmp <- tempdir()
unzip('data/SISmm2008_CMSAF.zip', exdir=tmp)
filesCMSAF <- dir(tmp, pattern='SISmm')
SISmm <- stack(paste(tmp, filesCMSAF, sep='/'))
## CM-SAF data is average daily irradiance (W/m2). Multiply by 24
## hours to obtain daily irradiation (Wh/m2)
SISmm <- SISmm * 24
\end{lstlisting}

The \texttt{RasterLayer} object with annual averages is computed from the
monthly means and stored using the native format of the \texttt{raster}
package.
\lstset{language=R,numbers=none}
\begin{lstlisting}
## Monthly irradiation: each month by the corresponding number of days
daysMonth <- c(31, 29, 31, 30, 31, 30, 31, 31, 30, 31, 30, 31)
SISm <- SISmm * daysMonth / 1000 ## kWh/m2
## Annual average
SISav <- sum(SISm)/sum(daysMonth)
writeRaster(SISav, file='SISav')
\end{lstlisting}

\section{Land Cover and Population Rasters}
\label{sec-4}

The NASA's Earth Observing System (EOS)\footnote{\url{http://eospso.gsfc.nasa.gov/}} is a coordinated
series of polar-orbiting and low-inclination satellites for
long-term global observations of the land surface, biosphere, solid
Earth, atmosphere, and oceans. NEO-NASA\footnote{\url{http://neo.sci.gsfc.nasa.gov}}, one of projects
included in EOS, provides a repository of global data imagery. We
use the population density and land cover classification
rasters. Both rasters must be downloaded from their respective
webpages as Geo-TIFF files.

\lstset{language=R,numbers=none}
\begin{lstlisting}
library(raster)
## http://neo.sci.gsfc.nasa.gov/Search.html?group=64
pop <- raster('875430rgb-167772161.0.FLOAT.TIFF')
## http://neo.sci.gsfc.nasa.gov/Search.html?group=20
landClass <- raster('241243rgb-167772161.0.TIFF')
\end{lstlisting}




%%% Local Variables:
%%% TeX-master: "../main.tex"
%%% End: