


Recently I found a post at \url{http://flowingdata.com/2011/05/11/how-to-map-connections-with-great-circles/} with a detailed tutorial to
map connections with \url{http://en.wikipedia.org/wiki/Great_circle} with R. I really liked the maps
of Facebook but, unfortunately, the code was not available.  The
tutorial of FlowingData is excellent, but I feel more comfortable
with the \url{http://cran.r-project.org/web/packages/sp} classes and methods, and with the \url{http://lattice.r-forge.r-project.org/} and
\url{http://latticeextra.r-forge.r-project.org/} packages.  Besides, I want to use the free spatial
data available from the \url{http://www.diva-gis.org/Data} project, from the developer of
the \url{http://cran.r-project.org/web/packages/raster} and the \url{http://cran.r-project.org/web/packages/geosphere/} packages.

Here is what I have got.

First, let's load the packages.


\lstset{language=R}
\begin{lstlisting}
library(lattice)
library(latticeExtra)
library(maps)
library(geosphere)
library(sp)
library(maptools)
library(raster)
library(rasterVis)
\end{lstlisting}

Now it's time to get the data. First, airports and flights:


\lstset{language=R}
\begin{lstlisting}
airports <- read.csv("http://datasets.flowingdata.com/tuts/maparcs/airports.csv", header=TRUE)
flights <- read.csv("http://datasets.flowingdata.com/tuts/maparcs/flights.csv", header=TRUE, as.is=TRUE)
\end{lstlisting}

With this information, following the code from the FlowingData
post, let's define a list of \texttt{SpatialLines}.  The \texttt{makeLines}
function defines a \texttt{SpatialLines} for each connection, and stores
the number of flights in the ID slot.  The \texttt{linesAA} is a list
with the result of \texttt{makeLines} over fsub with the \texttt{apply}
function.  Previously, the \texttt{fsub} data.frame has been ordered by
\texttt{cnt}, as the FlowingData post teaches.


\lstset{language=r}
\begin{lstlisting}
makeLines <- function(x){
  air1 <- airports[airports$iata == x[2],]##start
  air2 <- airports[airports$iata == x[3],]##end
  inter <- gcIntermediate(c(air1[1,]$long, air1[1,]$lat),
                          c(air2[1,]$long, air2[1,]$lat),
                          n=100,
                          sp=TRUE, addStartEnd=TRUE)
  inter@lines[[1]]@ID <- as.character(x[4])##cnt
  inter 
}

fsub <- flights[flights$airline == "AA",]
fsub <- fsub[order(fsub$cnt),]
linesAA <- apply(fsub, 1, makeLines)
\end{lstlisting}

Each component of the list can be plot with the \texttt{sp.lines}
function.  The \texttt{colIndex} function assigns a color from a palette
to the values of cnt stored in the ID slot.


\lstset{language=r}
\begin{lstlisting}
colIndex <- function(x, cnt, palette)palette[match(x, cnt)]
makePlot <- function(x, cnt, palette){
  idx <- as.numeric(x@lines[[1]]@ID)
  sp.lines(x, col.line=colIndex(idx, cnt, palette))
}
\end{lstlisting}

I define the palette for the list of lines with a combination of
\texttt{colorRampPalette}, \texttt{brewer.pal} and \texttt{level.colors}, following the
example of the help file of this last function.


\lstset{language=R}
\begin{lstlisting}
cnt <- fsub$cnt
breaks <- do.breaks(range(cnt), 30)
palLines <- colorRampPalette(brewer.pal('Greens', n=9))
colors <-level.colors(cnt,
                      at = breaks,
                      col.regions = palLines)
\end{lstlisting}

Let's get the elevation data and the boundaries:


\lstset{language=R}
\begin{lstlisting}
old <- setwd(tempdir())
download.file('http://www.diva-gis.org/data/msk_alt/USA_msk_alt.zip', 'USA_msk_alt.zip')
unzip('USA_msk_alt.zip')
download.file('http://www.gadm.org/data/shp/USA_adm.zip', 'USA_adm.zip')
unzip('USA_adm.zip')

prj <- CRS("+proj=longlat +datum=WGS84")

elevUSA <- raster('USA1_msk_alt.grd', projection=prj) ## Mainland
mapUSA <- readShapeLines('USA_adm1.shp', proj4string=prj)

setwd(old)
\end{lstlisting}


Now it's time for joining all together. I use the \texttt{layer} function from \texttt{latticeExtra} to draw the boundaries (with \texttt{sp.lines}) and the list of lines (with \texttt{lapply} and \texttt{makePlot}).


\lstset{language=R}
\begin{lstlisting}
library(colorspace)
myTheme <- rasterTheme(region=terrain_hcl(15))

levelplot(elevUSA,
          par.settings=myTheme,
          margin=FALSE, colorkey=FALSE) +
  layer(sp.lines(mapUSA, col.line='black', lwd=0.5)) +
  layer(lapply(linesAA, makePlot, cnt, colors))
\end{lstlisting}

\includegraphics[width=.9\linewidth]{figs/greatCircle.pdf}

